\citet{motolinia-reel-book2026} es el estudio más ambiciosos de la reforma reeleccionista en México. Fiel a la literatura norteamericana, se enfoca en la reelección legislativa, no en el Congreso federal sino en los subnacionales. El enfoque le permite aprovechar los calendarios diferenciados de entrada en vigor de la reforma entre los estados para afinar su estrategia empírica. Recabó evidencia monumental de índole diversa sobre los diputados locales (más de medio millón de discursos en la tribuna, diez mil votaciones nominales, miles de pertenencias a comisiones ordinarias y los presupuestos de gestoría de miles de legisladores). Su análisis arrojó resultados, en el mejor de los casos, mixtos. 

Comparado con los demás, los diputados que buscaron reelegirse tuvieron el doble de probabilidad de integrarse a comisiones con jurisdicción sobre políticas particularistas (\emph{pork}); hicieron 2 por ciento más menciones a proyectos tipo \emph{pork} para el distrito desde la tribuna; y reportaron presupuestos de gestoría 15 por ciento mayores. Las métricas de unidad partidista, en cambio, no manifiestan diferencias. 

Estos efectos son atribuibles a la reforma, pero sorprende que sean tan pequeños. 
